\documentclass{article}

\usepackage{amsmath}

\title{SIR Model Stability}
\author{Charlotte Maryjean White}

\begin{document}
	\maketitle
	
	\section{Model}
	
	We can begin with a simplified SIR (Susceptible-Infected-Removed) model with no births or deaths.
	
	\[ \dfrac{dS}{dt} = -\frac{\alpha}{N} SI \]
	\[ \dfrac{dI}{dt} = \frac{\alpha}{N} SI - \gamma I \]
	\[ \dfrac{dR}{dt} = \gamma I \]
	
	Where $\alpha$ is the contact rate, and $\gamma$ is the recovery rate. The initial conditions would be a constant population, N, and a positive, non-zero $S_0$ and $I_0$ and non-negative $R_0$.
	
	\[ S_0 + I_0 + R_0 = N \]
	\[ S_i + I_i + R_i = N \]
	\[ S_0 > 0, I_0 > 0, R_0 >= 0 \]
	
	In such a model with a closed population, there will be no endemic equilibrium. The disease would die out as the susceptible population is eventually depleted. Thus a more useful model to examine is the SIR model with births and deaths:
	
		
	\begin{thebibliography}{00}
		\bibitem{BrauerEtal2019} Brauer, F., Castillo-Chavez, C., Feng, Z. (2019). Introduction: A Prelude to Mathematical Epidemiology. In: Mathematical Models in Epidemiology. Texts in Applied Mathematics, vol 69. Springer, New York, NY. https://doi.org/10.1007/978-1-4939-9828-9\_1
		\bibitem{Allen1994} Allen, L.J.S. (1994). Some discrete-time SI, SIR, and SIS epidemic models. Mathematical Biosciences 124(1), 83-105. https://doi.org/10.1016/0025-5564(94)90025-6
		\bibitem{Strogatz2001} Strogatz, S. (2001). Nonlinear dynamics and chaos: with applications to physics, biology, chemistry and engineering. Perseus Books Group.
	\end{thebibliography}
	
\end{document}